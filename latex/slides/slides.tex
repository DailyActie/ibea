\documentclass[12pt]{beamer}
\usepackage{framed}
\usepackage{amsmath} 
\usepackage{amssymb}
\usepackage{bm}
\usepackage{dsfont}
\usepackage{appendixnumberbeamer}
\usepackage{algorithm}
\usepackage{algorithmic}

\usepackage[utf8]{inputenc}
\usepackage[T1]{fontenc}
%\usepackage{tgheros}
%\renewcommand*\familydefault{\sfdefault} %% Only if the base font of the document is to be sans serif
\usetheme{metropolis}
\usecolortheme{default}
\usefonttheme[onlymath]{serif}
%\beamertemplatenavigationsymbolsempty
%\hypersetup{pdfpagemode=UseNone} % don't show bookmarks on initial view

% a few macros
\newcommand{\bi}{\begin{itemize}}
\newcommand{\ei}{\end{itemize}}
\newcommand{\ig}{\includegraphics}

\title{Indicator-based evolutionary Algorithm}
\author{Karim Kouki \and Ahmed Mazari \and Daro Ozad \and Mihaela Sorostinean \and Aris Tritas}

\institute{
	M.Sc. Machine Learning, Information and Content - 
	University of Paris-Saclay
}
\date{\today}
\subject{Optimization}

\begin{document}

  \begin{frame}
  	\titlepage
  \end{frame}
  
  \begin{frame}
    \frametitle{Outline}
    \bi
    \item Multi-objective optimization
    \item IBEA
    \item Implementation \& Tests
    \item Conclusion
    \ei
    
  \end{frame}
  
    \begin{frame}
    \frametitle{Recombination}
    
    Different recombination operators.
    
    Choosing a recombination probability.
    
  \end{frame}
  
    \begin{frame}
    \frametitle{Mutation}
    Adapting the step-size is essential to find an optimal domain on each dimension.
    
    Fixing the mutation probability
  \end{frame}

  \begin{frame}
    \frametitle{Simulated Binary Crossover}
    \textbf{Idea:} control the domain of space in which offspring is generated.
    
    \textbf{Definition:} Approximate distribution to a high-probability stationary `spread` distribution with contracting and expanding distributions
    $$ c(\beta) = 0.5(n_c +1)\beta^{n_c}, \beta \leq 1$$
        $$ e(\beta) = 0.5(n_c +1)\frac{1}{\beta^{n_c+2}}, \beta > 1$$
        
    $\implies$ Tuning the distribution index.
  \end{frame}
  
  \begin{frame}
  \centering{
  Thank you!
  
  Questions?
  }
  \end{frame}
  
    
  
\begin{frame}[allowframebreaks]
  \frametitle<presentation>{Further Reading}    
\begin{thebibliography}{}
\beamertemplatearticlebibitems
\bibitem[Zitzler and Künzli]{ibea} Eckart Zitzler and Simon Künzli, “Indicator-Based Selection in Multiobjective Search”. In Parallel Problem Solving from Nature (PPSN 2004), pp. 832-842, 2004.
\bibitem[Kalyanmoy and Agrawal]{sbx} Deb, Kalyanmoy, and Ram B. Agrawal. "Simulated binary crossover for continuous search space." Complex Systems 9.3 (1994): 1-15.
\end{thebibliography}
\end{frame}
\end{document}